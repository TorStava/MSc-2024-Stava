
\chapter{Related Work}
\label{ch:related}

\instructions{Page budget for Related Work: 10 - 15 pages
%
\begin{itemize}
    \item Discuss the relevant literature related with your problem, in a coherent way to demonstrate that you have given due consideration to know: your problem and the possible sub-problems of interest in this work, the previous approaches, and their benefits and missing points or drawbacks.
    \item It is not about concatenating summaries of papers. Rather, it is about to discuss the problem, its details and challenges (difficulties and/or opportunities) you focus on, all this by relying on previous work, comparing what some did, what some else improved or added, what is missing, and so on.
    \item Moreover, mind to keep the focus in the areas you care during your discussion (since you control the story), and to remark what you consider useful, interesting, or challenging.
    \item In some cases it could be necessary and/or natural to have, first, an introduction (which could discuss works on more general aspects of the problem), and then split the rest of the chapter in sections, each discussing a particular aspect of the problem.
    \item If you define a new task, or create or improve a method, or develop a test collection, or perform a missing major evaluation, explain why it is interesting, why is different or helpful versus previous works.
    \item Mind that the reader may have never heard about these things. You need to discuss them in such detail that it is possible to follow later parts of the thesis without having to consult external resources.
    \item Always use natbib!
    \item Use \texttt{\textbackslash\.citet\{\}} for textual citation. For example, ``\citet{Balog:2018:Book} proposed...''
    \item Use \texttt{\textbackslash\.citep\{\}} for parenthetical citation. For example, ``In \citep{Zhang:2020:KDD} the idea of ..''
    \item Here is an example of a PhD thesis:  \citet{Maxwell:2019:PhDThesis} 
    \item Here is how a Journal article would look in the Bibliography: \citet{Sanderson:2010:FnTIR}
    \item Never write out Smith et al., there is a \texttt{\textbackslash\.citeauthor{}\{\}} command for that (but most likely what you're looking for is actually \texttt{\textbackslash\.citet\{\}}.
    \item Check the corresponding Bib entry, for citing online documentation~\citep{Rasa:2022:doc}; if you just need to include an URL of a website/code/dataset, use a footnote instead.\footnote{\url{https://rasa.com}}
\end{itemize}
}
