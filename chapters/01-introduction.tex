
\chapter{Introduction}
\label{ch:intro}

% \instructions{Page budget for Introduction: 3-5 pages}

\section{Background and Motivation}
\label{sec:intro:background}

% \instructions{
% \begin{itemize}
%     \item Awaken the reader's interest and convince her why the theme is important.
%     \item Background information might be historical in nature, or it might refer to previous research or practical considerations.
%     \item Provide an example or use case for the problem.
%     \item It should be written on a level that it's understandable by anyone with a computer science master's degree.
%     \item It might contain a small handful of citations if it is needed to justify some main claims or assumptions, but this is not the part to detail any related work and/or compare among each other.
% \end{itemize}
% }

Converting research findings into presentations is crucial in academic and professional settings. However, it can be quite challenging. This process transforms detailed research papers into clear, engaging presentations catering to various audiences, ranging from conference attendees to students. It demands considerable time, profound subject knowledge, and expertise in information design and storytelling skills.

The goal is to create a system that automatically generates academic paper presentations. This system will improve efficiency and bridge the gap between detailed research and audience-friendly presentations. By automating this process, researchers and professionals will have more time and cognitive resources to focus on their primary work. Moreover, such a system will ensure their findings reach a wider audience.

Automating the process of creating academic presentation slides involves several challenges. The system must accurately interpret the complex language used in academic papers, often filled with specific terminology and data. It should be able to identify the key points in a paper, understand the significance of different findings, and translate this information into visually appealing slides. Additionally, it must be able to adapt to the conventions followed by various research fields, which have unique ways of presenting and arguing their findings.

Significant advancements in natural language processing, information retrieval, machine learning, graph databases, and vector databases are required to tackle the difficulties faced in academic research. Utilizing these technologies makes it possible to process and visually represent the complex content of academic papers. Developing such a system would considerably enhance the automation of knowledge dissemination, making it easier for academics and professionals to convert research papers into presentations. This, in turn, will increase the accessibility and impact of their work, providing greater opportunities for future research and collaborations.

\section{Objectives}
\label{sec:intro:objectives}

% \instructions{
% \begin{itemize}
%    \item Define the goals of your study. It might be presented as a bullet list.
%   \item Structure your goal by Research Questions (RQs). Describe each of the problems to address in your work and formulate for it a clear research question.``The problem of bla is about$\dots$, We formulate the following RQ1 Can we provide a method for bla such that bla bla''
% \end{itemize}
% }

The main objective of this thesis is to create an innovative system that can generate presentations automatically from academic research papers. Specifically, the system will convert research papers in \LaTeX{} into presentations formatted in the \LaTeX{} Beamer framework. This project aims to make the process of sharing knowledge more efficient and accessible. To achieve this, the thesis will explore several critical research questions designed to address a distinct aspect of the system's development process.

\begin{itemize}
  \item \textbf{RQ1: Dataset Generation} – The first research question aims to determine whether creating a comprehensive dataset that includes research papers and their corresponding presentations, formatted according to LaTeX and Beamer specifications, is possible. This is a crucial step towards developing the proposed system, as it investigates the availability of source material needed for training and testing. The process will involve identifying potential sources for such documents, curating the dataset, and implementing strategies to ensure that it is diverse and representative.
  
  \item \textbf{RQ2: Information Extraction} – The second question deals with the methods to extract relevant information from research papers. This involves creating or using natural language processing techniques to identify and isolate important points, findings, and data within the complex structure of academic papers. The challenge here is two-fold: accurately discerning the content's significance among the academic jargon and structurally diverse formats of research papers.
  
  \item \textbf{RQ3: Information Summarization} – This question continues the previous one. It's about summarizing extracted information in a shorter form while retaining the main message. This summarization process is essential when presenting research findings on slides. The research will explore the algorithms and techniques used for effective summarization, ensuring that the most important information is highlighted while maintaining the research's integrity and context.
  
  \item \textbf{RQ4: Presentation Generation} – The final question seeks to establish a methodology for converting summarized information into coherent, high-quality presentations. This includes the textual content and the design and aesthetic aspects of the slides. The challenge is to automate the process of creating presentations that are not only informative but also visually appealing and engaging for the audience. This will require using design principles, information hierarchy, and customizable templates to create presentations that effectively communicate research findings.
\end{itemize}

This research aims to answer essential questions that will benefit academics and educators. It will provide a new tool to help them process and present documents automatically, which is currently challenging. The goal is to solve the time-consuming presentation creation process. Doing so will help enhance the accessibility and dissemination of scientific knowledge. Through rigorous investigation and development, this research will push the boundaries of what is possible in automatic document processing and presentation generation.

\section{Approach and Contributions}
\label{sec:intro:approach}

% \instructions{
% \begin{itemize}
%   \item Give a brief summary of your overall approach.
%   \item Summarize the specific contributions that you made in this thesis (e.g., a task definition, a method or model, a test collection, empirical results, analysis, etc.). It might be presented as a bullet list.
% \end{itemize}
% }

Converting a research paper into a presentation requires a complex process involving summarizing, extracting key elements, or a combination of both. This thesis presents a new system called TEX2BEAM, which automates the creation of presentation slides in \LaTeX{} Beamer format from research papers in \LaTeX{}. The TEX2BEAM system is designed to intelligently navigate through the dense academic text of a research paper, extract relevant information, and distill it into a clear and concise presentation format.

The operational framework of TEX2BEAM is based on a set of advanced algorithms and methodologies designed to comprehend the intricate structure and content of academic papers. At first, the system uses natural language processing (NLP) techniques to analyze the text. It identifies and classifies various sections of the paper, such as the abstract, introduction, methodology, results, and conclusion. This step is pivotal in identifying the parts of the text that are most likely to contain crucial insights and findings relevant to a summarized presentation.

TEX2BEAM uses extractive and abstractive summarization techniques to create a presentation from a research paper. In the extractive phase, the system selects verbatim snippets representing the text's main ideas and findings. Then, in the abstractive phase, these ideas are rephrased and condensed into more concise statements tailored for slide-based presentations. This approach ensures that the presentation effectively captures the essence of the research paper while being engaging and accessible to the audience.

The TEX2BEAM system has a vital component in the form of an algorithm that organizes the summarized content into a well-structured and visually attractive presentation. This includes selecting the best sequence of slides and generating or choosing suitable visual aids like figures, tables, and charts. Moreover, the algorithm applies design principles to improve readability and viewer engagement.

The TEX2BEAM tool will be thoroughly tested to evaluate its effectiveness and efficiency. The tests will measure its ability to extract relevant information from research papers and create compelling presentations. The evaluation will include metrics such as the accuracy of information extraction, the clarity and conciseness of the summarization, and the overall quality of the generated presentations. This evaluation will demonstrate the potential of TEX2BEAM as a valuable tool for academics and professionals, making creating presentations easier and facilitating the sharing of research findings.


\section{Outline}
\label{sec:intro:outline}

% \instructions{
% \begin{itemize}
%     \item Give an overview of the main points and the structure of your thesis. ``Chapter 2 covers ... Chapter 3 describes ... ''
%     \item Show in a natural way how the different parts (chapters) relate to each other.
% \end{itemize}
% }

This thesis has been written to present a logical and comprehensive research account. It covers all aspects of the research, from reviewing related studies to concluding remarks on the findings and prospects. Each chapter has been carefully crafted to build upon the previous one, taking the reader on a journey of developing the TEX2BEAM system, an innovative solution for automatically generating presentations from research papers.

\textbf{Chapter 2} explores the existing body of work relevant to the thesis's core objectives. It delves into the methodologies and technologies used to summarize texts, extract key information, and generate presentations from academic documents and reports. This chapter aims to position the current research within the broader academic discourse by highlighting the advancements in the field and identifying the gaps that the TEX2BEAM system aims to address.

\textbf{Chapter 3} describes the methodology and technical framework used to develop the TEX2BEAM system. It explains the architectural design, algorithms, and computational strategies employed to process research papers written in \LaTeX{} format and convert them into informative and visually appealing presentations in \LaTeX{} Beamer format. The approach incorporates novel features, including the system's ability to parse academic language intelligently, identify significant content, and summarize and represent this information effectively in a presentation-ready format.

\textbf{Chapter 4} is focused on evaluating the performance of the TEX2BEAM system. It presents a detailed analysis of the system's accuracy in extracting information, effectiveness in summarizing content, and the overall quality and coherence of the generated presentations. The evaluation process includes both quantitative measures and qualitative feedback, providing a comprehensive understanding of the system's capabilities, limitations, and potential impact on the field.

\textbf{Chapter 5} concludes the thesis and summarizes the key insights and contributions made through the research. It highlights the significance of automating the process of generating presentations from research papers and discusses the implications for academic and professional dissemination of knowledge. Furthermore, this chapter outlines potential areas for future research, proposing ways in which the TEX2BEAM system could be improved, expanded, or used in new areas.
