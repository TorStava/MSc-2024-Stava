
\chapter{Introduction}
\label{ch:intro}

% \instructions{Page budget for Introduction: 3-5 pages}

\section{Background and Motivation}
\label{sec:intro:background}

% \instructions{
% \begin{itemize}
%     \item Awaken the reader's interest and convince her why the theme is important.
%     \item Background information might be historical in nature, or it might refer to previous research or practical considerations.
%     \item Provide an example or use case for the problem.
%     \item It should be written on a level that it's understandable by anyone with a computer science master's degree.
%     \item It might contain a small handful of citations if it is needed to justify some main claims or assumptions, but this is not the part to detail any related work and/or compare among each other.
% \end{itemize}
% }

The process of creating a presentation from a research paper can be a time consuming and tedious task. The goal of this project is to automate this process, by creating a system that can automatically generate presentations from research papers.


\section{Objectives}
\label{sec:intro:objectives}

% \instructions{
% \begin{itemize}
%    \item Define the goals of your study. It might be presented as a bullet list.
%   \item Structure your goal by Research Questions (RQs). Describe each of the problems to address in your work and formulate for it a clear research question.``The problem of bla is about$\dots$, We formulate the following RQ1 Can we provide a method for bla such that bla bla''
% \end{itemize}
% }

The main objective of this thesis is to create a system that can automatically generate presentations from research papers. In detail, the research papers will be in \LaTeX{} format and the presentations will be in \LaTeX{} Beamer format. In order to achieve this objective, the following research questions will be addressed:

\begin{itemize}
  \item RQ1: Can we generate a dataset of research papers and their presentation in the required \LaTeX{} and Beamer formats?
  \item RQ2: Can we provide a method for extracting the relevant information from a research paper?
  \item RQ3: Can we provide a method for summarizing the extracted information?
  \item RQ4: Can we provide a method for generating a coherent and high-quality presentation from the summarized information?
\end{itemize}

\section{Approach and Contributions}
\label{sec:intro:approach}

% \instructions{
% \begin{itemize}
%   \item Give a brief summary of your overall approach.
%   \item Summarize the specific contributions that you made in this thesis (e.g., a task definition, a method or model, a test collection, empirical results, analysis, etc.). It might be presented as a bullet list.
% \end{itemize}
% }

Creating a presentation from a research paper can be viewed as either a summarization task or an extractive task, or a combination of both. We propose a novel method, \textbf{REP2BEAM} of generating presenation slides in \LaTeX{} Beamer format from research papers in \LaTeX{} format. The system will extract the relevant information from the research paper and then summarize it in a presentation. The system will be evaluated on how well it can extract the relevant information from the paper and how well it can summarize it in a presentation.


\section{Outline}
\label{sec:intro:outline}

% \instructions{
% \begin{itemize}
%     \item Give an overview of the main points and the structure of your thesis. ``Chapter 2 covers ... Chapter 3 describes ... ''
%     \item Show in a natural way how the different parts (chapters) relate to each other.
% \end{itemize}
% }

The rest of this thesis is structured as follows. Chapter~\ref{ch:related} discusses relevant work related to the tasks of summarizing, extracting, and generating presentations based on documents and reports. Chapter~\ref{ch:approach} describes in detail the approach developed in this thesis in order to create the Automatic Generation of Presentations for Research Papers (\textbf{AGPRP})/\textbf{REP2BEAM} system (naming to be decided). Chapter~\ref{ch:eval} evaluates the results of the AGPRP system by looking av various metrics. Chapter~\ref{ch:conclusion} wraps up and concludes the thesis.